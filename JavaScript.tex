\documentclass[10pt,a4paper]{article}
\usepackage[T1]{fontenc}
\usepackage[utf8]{inputenc}
\usepackage[spanish,es-tabla]{babel}
\parindent=0cm %Modificar tamaño de sangria
\usepackage{amsmath}
\usepackage{amssymb,amsfonts,latexsym,cancel}
\usepackage{graphicx}
\usepackage{epstopdf}
\usepackage{float}
\usepackage{subfigure}
\usepackage{array}
\usepackage{longtable}
\newcolumntype{E}{>{$}c<{$}}
\setcounter{MaxMatrixCols}{40}
\usepackage{bm}
\usepackage{xcolor}
%%%%%%%%%%%%%%%%%%%%%%%%%%%%%%%%%%%%%
%%%PAQUETES O CONFIGURACION NUEVA%%%%
%%%%%%%%%%%%%%%%%%%%%%%%%%%%%%%%%%%%%
\usepackage[lmargin=2cm, rmargin=2cm,top=2.5cm,bottom=2cm]{geometry}
\usepackage{fancyhdr}
\pagestyle{fancy}
\fancyhead{}%%Es para limpiar el documento
\fancyhead[C]{Titulo de nuestro articulo}
\fancyhead[R]{\includegraphics[scale=0.07]{figuras/logo}}
\fancyfoot{}
\fancyfoot[R]{\thepage}
\fancyfoot[L]{Bryan Ricardo}
\renewcommand{\headrulewidth}{0.9pt}
\renewcommand{\footrulewidth}{0.5pt}
%%%%%%%%%%%%%%%%%%%%%%%%%%%%%%%%%%%%%
\begin{document}
\begin{titlepage}
\begin{center}
\vspace*{2\baselineskip}%%saltos de linea
\hrule height 3pt
\vspace*{0.5\baselineskip}%%saltos de linea
{\Huge \textbf{Universidad Autonoma de Aguascalientes}}
{\Large \textbf{LICENCIATURA EN MATEMATICAS APLICADAS}}
\vspace*{0.5\baselineskip}%%saltos de linea
\hrule
\vspace*{0.5\baselineskip}%%saltos de linea
\includegraphics[scale=0.5]{figuras/logo}
\vspace*{2\baselineskip} \\%%saltos de linea
\textbf{\large MATERIA: Java Script} \\
\vspace*{1.5\baselineskip}
\textbf{\large Docente: Bryan Ricardo Barbosa Olvera} \\
\vspace*{1.5\baselineskip}
\textbf{\large FECHA DE CREACION: 18 de Junio de 2022} \\  
\vspace*{3\baselineskip}
\includegraphics[scale=0.5]{figuras/imagen}
\vfill
BRYAN RICARDO BARBOSA OLVERA \\
\today \\

\end{center}
\end{titlepage}

\section{POO}
\subsection{Crear Objetos}
const producto  $ = \{ $  \\ nombre: "Monitor 20 pulgadas", \\ precio: 300, \\  $ \} $ 
\subsection{Acceder a los valores}
const producto  $ = \{ $  \\ nombre: "Monitor 20 pulgadas", \\ precio: 300, \\  $ \} $ \\ 
console.log(producto.nombre);\\
\subsection{Agregar O Eliminar valores} 
const producto  $ = \{ $  \\ nombre: "Monitor 20 pulgadas", \\ precio: 300, \\  $ \} $ \\ 
//Agregar  nuevas propiedades \\
producto.imagen = "imagen.jpg"; \\
//ELiminar propiedades del objeto \\
delete producto.precio \\
\subsection{Destructuring} 
const producto  $ = \{ $  \\ nombre: "Monitor 20 pulgadas", \\ precio: 300, \\  $ \} $ \\ 
//Crea y asigna el valor a la variable \\
const {nombre, precio} = producto; \\
console.log(nombre,precio);\\
\subsection{Destructuring de objetos anidados} 
const producto = $ \{ $ \\ nombre: "Monitor 20 pulgadas", \\ precio: 300, \\ informacion: $ \{ $ \\
fabricacion:  $ \{ $ \\ pais: 'CHina' \\ $ \} \} \}$  \\
const $\{ $ nombre, informacion , informacion : $ \{ $ fabricacion: $ \{ $pais $ \} \} \} = $ producto;\\
console.log(pais);\\
\newpage
\subsection{Conjelar un objeto} 
$ " $use strict $ " $ \\
const producto  $ = \{ $  \\ nombre: "Monitor 20 pulgadas", \\ precio: 300, \\  $ \} $ \\ 
//Al conjelar el objeto no deja que se le agregen o eliminen valores del objeto \\
Object.freeze(producto);\\
//el siguiente comando indica con un true si esta conjelado el objeto\\
console.log(Object.isFrozen(producto));\\
\subsection{Sellar un objeto}
$ " $use strict $ " $ \\
const producto  $ = \{ $  \\ nombre: "Monitor 20 pulgadas", \\ precio: 300, \\  $ \} $ \\ 
//Al sellar un objeto es parecido a conjelarlo con la diferencia que le permite cambiar el valor de las llaves \\
Object.seal(producto);\\
producto.precio = 200; \\
//el siguiente comando indica con un true si esta sellado el objeto\\
console.log(Object.isSealed(producto));\\
\subsection{Spread Operator o Rest Operator}
const producto  $ = \{ $  \\ nombre: "Monitor 20 pulgadas", \\ precio: 300, \\  $ \} $ \\ 
const medidas $ = \{ $ \\ peso: '1kg', \\ medida: '1m' \\ $ \} $ \\
//Lo que realiza es unir dos objetos en uno solo \\
const resultado = $ \{ $ ...producto , ...medidas  $ \} $ \\
\subsection{La palabra reservada this}
//Te permite no perder la referencia de la varable que se esta llamando y no tomar variables fuera del objeto
const producto  $ = \{ $  \\ nombre: "Monitor 20 pulgadas", \\ precio: 300, \\  ,mostrarInfo: function $ \left(  \right) \{$ \\
console.log(`EL producto tiene como nombre: $ \$ \{ $ this.nombre  $ \} $`) $ \} $ \\ $ \} $ \\ 
\newpage
\subsection{.keys .values .entries} 
const producto  $ = \{ $  \\ nombre: "Monitor 20 pulgadas", \\ precio: 300, \\  $ \} $ \\ 
//.keys te retorna las llaves del objeto en un objeto \\
console.log(Object.keys(producto)); \\
//.values te retorna los valores del objeto en un objeto \\
console.log(Object.values(producto)); \\
//.entries te retorna las llaves y los valores del objeto en pares en  un objeto \\
console.log(Object.entries(producto)); \\

\newpage
\section{ARRAYS}
\subsection{.keys .values .entries} 


\end{document}
